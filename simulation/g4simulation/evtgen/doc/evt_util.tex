\section{The routines {\tt decay\_angle} and {\tt decay\_angle\_chi}}
\label{sect:evtutil}

This section will describe the utility routines, 
{\tt EvtDecayAngle}, {\tt EvtDecayPlaneNormalAngle}, 
and  {\tt EvtDecayAngleChi},
that compute decay angles.  The function {\tt EvtDecayAngle} is 
useful for finding the decay angle, known as the helicity
angle, in a decay tree that includes a two body decay.
Teh routine {\tt EvtDecayPlaneNormalAngle} calculates the angle
of the decay plane normal in a three-body decay.
{\tt EvtDecayAngleChi} computes the azimuthal angle between two decay
planes in sequencial decays such as $B \rightarrow D^* D^*$ or
$B \rightarrow D^* \ell \nu$.

To call {\tt EvtDecayAngle}, the syntax is 
\begin{verbatim} 
costheta=EvtDecayAngle(P,Q,D),
\end{verbatim}
where {\tt P}, {\tt Q} and {\tt D} are of type {\tt EvtVector4R}.
This routine returns the cosine of the angle $\theta$, as defined in
Figure~\ref{fig:decay_angle}. The decay angle calculated is that between
the flight direction of the 
daughter meson, {\tt D}, in the rest frame of {\tt Q} (the parent of {\tt D}),
with respect to {\tt Q}'s
flight direction in {\tt P}'s (the parent of {\tt Q})
rest frame. {\tt P}, {\tt Q}, and {\tt D}
are the momentum four vectors of these particles in any frame of reference.
The decay angle is
computed using the (manifestly invariant) expression
\begin{equation}
\cos\theta = { {(P\cdot D) M^2_Q- (P\cdot Q)(Q\cdot D)} \over
        	      \sqrt{[(P\cdot Q)^2-M^2_QM^2_P]
		[(Q\cdot D)^2-M^2_QM^2_D]}}.
\end{equation}



To call {\tt EvtDecayPlaneNormalAngle} the syntax is 
\begin{verbatim} 
costheta=EvtDecayPlaneNormalAngle(P,Q,D1,D2),
\end{verbatim}
where {\tt P}, {\tt Q}, {\tt D1}, and {\tt D2} are of type {\tt EvtVector4R}.
This routine returns the cosine of the angle $\theta$ of the normal to the
decay plane.
 The angle calculated is that between
the normal of the decay plane formed by the 
daughter mesons, {\tt D1} and {\tt D2}, in the rest frame of {\tt Q} (the parent of {\tt D1} and {\tt D2}),
with respect to {\tt Q}'s
flight direction in {\tt P}'s (the parent of {\tt Q})
rest frame. {\tt P}, {\tt Q}, {\tt D1}, and {\tt D2}
are the momentum four-vectors of these particles in any frame of reference.
The decay angle is
computed using the (manifestly invariant) expression
\begin{equation}
\cos\theta = { M_QP\cdot L} \over
        	      \sqrt{[(P\cdot Q)^2-M^2_PM^2_Q]
		[-L^2]}
\end{equation}
where $L_{\nu}=\epsilon_{\nu\mu\alpha\beta}Q^{\nu}D_1^{\alpha}D_2^{\beta}$.




The routine {\tt EvtDecayAngleChi} is used to calculate the azimuthal 
angel, $\chi$, between the decay planes of a pair of two body
decays.
As illustrated in Figure~\ref{fig:decay_angle_chi},
$\chi$ is the
angle 
calculated from the particle {\tt d1} to the particle {\tt h1} as
seen from the direction of particle {\tt D}. The form of the call to this 
subroutine is 
\begin{verbatim} 
chi=EvtDecayAngleChi(p4_parent,p4_d1,p4_d2,p4_h1,p4_h2),
\end{verbatim} 

where {\tt p4\_d1}, {\tt p4\_d2}, {\tt p4\_t1}, {\tt p4\_h2}
are of type {\tt EvtVector4R} and are the four momenta
of the daughter particles as illustrated in
Figure~\ref{fig:decay_angle_chi}.


\begin{figure}[hbtp]
\centerline{\epsfig{figure=decay_angle.eps,height=3.5in,width=3.5in}}
\caption[Definition of the decay angle.]
{
Definition of the decay angle.
}
\label{fig:decay_angle}
\end{figure}

\begin{figure}[hbtp]
\centerline{\epsfig{figure=decay_angle_chi.eps,height=2.5in,width=4.5in}}
\caption[Definition of the $\chi$ angle.]
{
Definition of the $\chi$ angle.
}
\label{fig:decay_angle_chi}
\end{figure}










