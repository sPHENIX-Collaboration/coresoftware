\section{Lineshape determination (new 11/2002)}

\subsection{Introduction}

This note summarizes the proposed lineshape treatment
for SP5 Monte Carlo production in {\tt EvtGen}.  Some 
significant changes have been made since SP4 given 
input from users during the past year.  The generater
now works in several steps:
\begin{enumerate}
\item The complete decay chain is determined according
to the branching fractions given in {\tt EvtGen/DECAY.DEC}
and the user decay file.
\item The mass of each particle in the decay chain 
is determined, as described below.
\item The kinematics of each particle are determined 
according to the decay model specified.
\end{enumerate}

As these are separate steps, the algorithm for
the determination of masses is not as general
is would be possible if the mass determination and
kinematic determination were combined.  This would
entail a more significant code rewrite, which we
were not able to undertake.  

\subsection{Lineshapes}

\subsubsection{What is the default?}

The short answer is that it depends on the decay.  For a
particle $P$, the algorithm is as follows:

\begin{enumerate}
\item Something like 30\% of $B$ mesons are decayed
using JETSET. If $P$ is produced by JETSET, then
its lineshape is determined by JETSET.  JETSET uses
a nonrelativistic Breit-Wigner.
\item If $P$ or any of its daughters has a spin other
than 0, 1/2, 1, 3/2, or 2,
a nonrelativistic Breit Wigner is used.
\item If the decay $P \to D_1 D_2$ proceeds by
a higher partial wave than $D$ wave, a nonrelativistic
Breit Wigner is used. (in practice, this case appears
to never occur in our DECAY.DEC)
\item $G \to P P_1$ and $P \to D_1 D_2$ (ie, $P$ is
produced and decayed in a two body decay): A relativistic
Breit-Wigner is used.  The corresponding amplitude is
\begin{equation}
A(m) = {{(p_{G \to P P_1}^{L_{G \to P P_1}} )
(p_{P \to D_1 D_2}^{L_{P \to D_1 D_2}})} \over {m^2 -{m_0}^2 + im \Gamma(m)}} 
({B(p_{G \to P P_1}) \over B(p^{\prime}_{G \to P P_1})})
({B(p_{P \to D_1 D_2}) \over B(p^{\prime}_{P \to D_1 D_2})})
\label{eq:ls}
\end{equation}
where $m_0$ is the nominal particle mass as specified
in {\tt PDT/pdt.table} and $p$ is the daughter
momentum in the rest frame of its parent (for the
specified decay chain). $p^{\prime}$ is the same, but for the nominal
mass of P.
$\Gamma(m)$ is
\begin{equation}
\Gamma(m) = \left( {p \over p^{\prime}} \right)^{2L_{P \to D_1 D_2}+1}
\left( m_0 \over m \right) \Gamma_0 
\left[ B(p) \over B(p^{\prime}) \right]. 
\label{eq:gamma}
\end{equation}
Here $\Gamma_0$ is the nominal particle width has
specified in {\tt PDT/pdt.table}.  $p$ and $p^{\prime}$
are the daughter momenta in the rest frame of $P$, 
if $P$ has mass $m$ or
$m_0$, respectively.  $L$ is the angular momentum
of the decay specified in its subscript, which is assumed 
to be the lowest allowed given the parent and daughter spins.
$B(p)$ is the Blatt-Weisskopf form factor, which depends on
$L$:
\begin{eqnarray}
B(p) &=& 1  \nonumber \\
B(p) &=& {1 \over \sqrt{1+(Rp)^2}} \nonumber \\
B(p) &=& {1 \over \sqrt{1+{(Rp)^2 \over 3} + {(Rp)^4 \over 9}}} 
\end{eqnarray}
for $L_{P \to D_1 D_2}$ equal to 0, 1, or 2. $R$ is $3~{\rm GeV}^{-1}$.
\item $G \to P P_1 P_2 X$ and $P \to D_1 D_2$ (ie, $P$ has
two daughters): Very similar to the above case, except the
birth momentum factor and form factor are removed:
\begin{equation}
A(m) = {{
(p_{P \to D_1 D_2}^{L_{P \to D_1 D_2}})} \over {m^2 -{m_0}^2 + im \Gamma(m)}} 
({B(p_{P \to D_1 D_2}) \over B(p^{\prime}_{P \to D_1 D_2})})
\label{eq:ls2}
\end{equation}


\item $P \to P_1 P_2 P_3 X$ (ie, $P$ decays to more than
two daughters): A nonrelativitic Breit-Wigner is
used.

\end{enumerate}




\subsubsection{Mass cutoffs, etc}

Cutoffs are listed in {\tt PDT/pdt.table}.  The cutoff
given represents the maximum $\Delta$ allowed from the 
nominal mass, but only on the low side.  That is, the
lowest allowed mass is $m - \Delta$. The highest allowed
mass is $m + 15 \Gamma$.  In the case that the cutoff
is not given, the lowest mass is ${\rm max}(0, m-15 \Gamma)$.

\subsubsection{What can be changed in user decay files?}

We have added some flexibility in how lineshapes
can be generated for signal Monte Carlo.  We hope
that this will simplify systematic studies.  The
various options are listed below.  As with other
features of user decay files, be sure to test your
changes with {\tt GeneratorsQA} before submitted your
request for a Monte Carlo sample.  
\begin{itemize}
\item Lowest generated mass
\begin{verbatim}
ChangeMassMin rho0 0.7
\end{verbatim}
\item Highest generated mass
\begin{verbatim}
ChangeMassMax rho0 0.9
\end{verbatim}
\item Presence of the birth momentum factor and form factor
(${p_P}^{L_{G \to P P_1}}$ and corresponding form factor)
\begin{verbatim}
IncludeBirthFactor rho0 no
\end{verbatim}
\item Presence of the decay momenum factor and form factor
(${p_D}^{L_{P \to D_1 D_2}}$ and corresponding form factor)
\begin{verbatim}
IncludeDecayFactor rho0 no
\end{verbatim}
\item Blatt-Weisskopf factor (in $GeV^{-1}$)
\begin{verbatim}
BlattWeisskopf rho 3.0
\end{verbatim}
\item Lineshape used.  There are a few alternative
lineshapes available:
\begin{verbatim}
LSNONRELBW rho0
LSFLAT rho0
\end{verbatim}
Note that LSFLAT should be used together with the ChangeMassMin
and ChangeMassMax options.
\item Mass and width
\begin{verbatim}
Particle rho0 0.8 0.2
\end{verbatim}
\end{itemize}

Some of these are illustrated in Figures~\ref{fig:nom}
to~\ref{fig:flat} for the $\rho$ mass in the decay $B \to \rho \pi$ and $\rho
\to \pi \pi$.

\begin{figure}
\epsfig{figure=nominal.eps,width=0.9\linewidth}
\caption{The nominal $\rho$ mass distribution in 
$B \to \rho \pi$ and $\rho \to \pi \pi$.}
\label{fig:nom}
\end{figure}

\begin{figure}
\epsfig{figure=masswid.eps,width=0.9\linewidth}
\caption{The $\rho$ mass distribution in 
$B \to \rho \pi$ and $\rho \to \pi \pi$ having included {\tt Particle rho0 0.8 0.2} 
in the user decay table.}
\label{fig:masswid}
\end{figure}

\begin{figure}
\epsfig{figure=min7.eps,width=0.9\linewidth}
\caption{The $\rho$ mass distribution in 
$B \to \rho \pi$ and $\rho \to \pi \pi$ having included {\tt ChangeMassMin rho0 0.7} 
in the user decay table.}
\label{fig:min}
\end{figure}

\begin{figure}
\epsfig{figure=max9.eps,width=0.9\linewidth}
\caption{The $\rho$ mass distribution in 
$B \to \rho \pi$ and $\rho \to \pi \pi$ having included {\tt ChangeMassMax 0.9} 
in the user decay table.}
\label{fig:max}
\end{figure}

\begin{figure}
\epsfig{figure=nodecayp.eps,width=0.9\linewidth}
\caption{The $\rho$ mass distribution in 
$B \to \rho \pi$, $\rho \to \pi \pi$ using {\tt IncludeDecayFactor rho0 no} 
in the user decay table.}
\label{fig:decayfactor}
\end{figure}

\begin{figure}
\epsfig{figure=nobirthp.eps,width=0.9\linewidth}
\caption{The $\rho$ mass distribution in 
$B \to \rho \pi$, $\rho \to \pi \pi$ using {\tt IncludeBirthFactor rho0 no} 
in the user decay table.}
\label{fig:birthfactor}
\end{figure}

\begin{figure}
\epsfig{figure=nonrel.eps,width=0.9\linewidth}
\caption{The $\rho$ mass distribution in 
$B \to \rho \pi$ and $\rho \to \pi \pi$ having included {\tt LSNONRELBW rho0} 
in the user decay table.}
\label{fig:massnonrel}
\end{figure}

\begin{figure}
\epsfig{figure=flat.eps,width=0.9\linewidth}
\caption{The $\rho$ mass distribution in 
$B \to \rho \pi$ and $\rho \to \pi \pi$ having included {\tt LSFLAT rho} 
in the user decay table.}
\label{fig:flat}
\end{figure}








