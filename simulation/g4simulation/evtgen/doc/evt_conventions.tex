\section{Conventions}

This section discusses the conventions we use
for various physics quantities in the code.

\subsection{Units}

In EvtGen, $c=1$, such that 
mass, energy and momentum are all measured in units of GeV. 
Similarly, time and space have units of mm.


\subsection{Four-vectors}
There are two types of four-vectors used in EvtGen,
{\tt EvtVector4R} and {\tt EvtVector4C}, which are real and
complex respectively. 

A four-vector is represented by $p^{\mu}=(E,\vec{p})$.
When a four-vector is used its components
are always corresponding to raised indices. 
A contraction of
two vectors $p$ and $k$ ({\tt p*k})  automatically lowers 
the indices on $k$ according to the metric $g={\rm diag}(1,-1,-1,-1)$
so that {\tt p*p} is the mass squared of a particle with
four-momentum $p$.


\subsection{Tensors}

We currently only support complex second rank tensors. As
in the case of vectors, tensors are always reperesented with
all indices raised. The convention for the totaly antisymmetric
tensor, $\epsilon_{\alpha\beta\mu\nu}$, is $\epsilon_{0123}=+1$.

\subsection{Dirac spinors}
\label{sect:diracspinor}

Dirac spinors are represented as a 4 component spinor in the Dirac-Pauli representation,
with initial state fermions or final state anti-fermions. 

\subsection{Gamma matrices}

Dirac gamma matrices are also represented in the Dirac-Pauli
representation, which has

\begin{eqnarray}
\gamma^0=\left[\begin{array}{rrrr}
          1 & 0 & 0 & 0 \\
          0 & 1 & 0 & 0 \\
          0 & 0 & -1 & 0 \\
          0 & 0 & 0 & -1 \\
          \end{array}\right],& &
\gamma^1=\left[\begin{array}{rrrr}
          0 & 0 & 0 & 1 \\
          0 & 0 & 1 & 0 \\
          0 & -1 & 0 & 0 \\
         -1 & 0 & 0 & 0 \\
          \end{array}\right],\\
\gamma^2=\left[\begin{array}{rrrr}
          0 & 0 & 0 & -i \\
          0 & 0 & i & 0 \\
          0 & i & 0 & 0 \\
         -i & 0 & 0 & 0 \\
          \end{array}\right],& &
\gamma^3=\left[\begin{array}{rrrr}
          0 & 0 & 1 & 0 \\
          0 & 0 & 0 & -1 \\
          -1 & 0 & 0 & 0 \\
          0 & 1 & 0 & 0 \\
          \end{array}\right].
\end{eqnarray}
This gives
\begin{equation}
\gamma^5=i\gamma^0\gamma^1\gamma^2\gamma^3
	={i\over4!}\epsilon_{\lambda\mu\nu\pi}
\gamma^{\lambda}\gamma^{\mu}\gamma^{\nu}\gamma^{\pi}
=\left[\begin{array}{rrrr}
          0 & 0 & 1 & 0 \\	
          0 & 0 & 0 & 1 \\	
          1 & 0 & 0 & 0 \\
          0 & 1 & 0 & 0 \\
          \end{array}\right].
\end{equation}



